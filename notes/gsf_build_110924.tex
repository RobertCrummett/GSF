%---------------------------------
% Bible Study Notes
%
% Author: R Nate Crummett
%---------------------------------
\documentclass[12pt]{article}
\input{../bible}

\pagenumbering{gobble}
\setlength{\parindent}{0pt}

\begin{document}
	\textbf{Growing Stone Fellowship \hfill Notes}

	\hfill \today
	
	\section*{\normalsize Fasting and Prayer}

	\begin{quote}
		\BibleMatthewSeventeenSixteen{}
		\BibleMatthewSeventeenSeventeen{}
	\end{quote}
	
	Was Christ taling to the disciples? Or to to the people?

	\begin{quote}
		\BibleMatthewSeventeenNineteen{}
		\BibleMatthewSeventeenTwenty{}
		\BibleMatthewSeventeenTwentyOne{}
	\end{quote}

	Do we focus on fasting enough? Christ is making it evident that
	we need to fast and pray in order to deliver others. People's
	experience is that fastig is not as heavily practiced as prayer.
	Several days of fasting is something foreign to some people.

	Is Christ playing when he tells us any obstacle can be overcome
	with faith? His analogy is fantastic.

	The disciples thought they could cast the demon out, but were
	unsuccessful. He is saying that the reson it did not work was
	because of their lack of faith. Christ is saying somethings you
	will not overcome unless you pray and fast.

	From a textual critical standpoint, the oldest manuscripts
	fo not have the last phrase about prayer and fasting.

	\begin{quote}
		\BibleMarkTwoEighteen{}
		\BibleMarkTwoNineteen{}
	\end{quote}

	Jesus was God in the flesh. What the disciples could get from
	Christ was more than the disciples could get from John. Christ
	was capable was very much more. We fast and pray because we
	need more than our teachers have to offer. We need God.

	\begin{quote}
		\BibleJohnFourteenNine{}
	\end{quote}

	If Christ can breath on the disciples and give them power
	to heal and deliver others, you do not need God. He is God.

	Christ is letting the disciples know that there is a way that
	he is with the Father that we are also going to have to be.
	The disciples ability to deliver someone is in order to give
	glory to who the Christ is, and who his Father is. It is all
	about Christ.

	Church culture has taught that fasting is relating to letting go.
	Fasting may be better thought of as a method of asking God for
	more. The only thing the word directs us to ``let go of'' is 
	sin.

	\begin{quote}
		\BiblePsalmFiftyOneSeventeen{}
	\end{quote}

	\begin{quote}
		\BibleIsaiahFiftySevenFifteen{}
	\end{quote}

	\begin{quote}
		\BibleIsaiahSixtySixTwo{}
	\end{quote}

	We can pray and fast not only because things get bad, but
	also because we need to tap into the energy. Praying and fasting
	is a way to thank God, it is a way to direct your focus towards
	the Father's buisness. Chirst fasted and prayed, yet was
	without sin. He did not need to empty himself of any sin, yet
	he was disciplined.

	The discipline behind fasting and prayer is a way to focus
	ourselves. This is part of the training for when you get into
	the field. \textbf{Fasting and prayer is part of staying ready.}
	You need to ask to be strong, to be like a damn warrior.

	We talk to stay connected. We pray to God to stay connected with
	God. The fasting seems to be an aspect of this connection. By
	connection we mean relationship. When Christ is with the disciples
	in the flesh, the disciples already have a relationship with God
	in the flesh. But when we leaves, to keep that relationship prayer
	and fasting becomes more importance.

	Why did Christ have to fast for forty days? When the flesh is 
	made weak the spirit is made strong. Chirst was preparing for 
	ministry.

	\section*{\normalsize What does it mean to see God's Kingdom?}

	\begin{quote}
		\BibleMatthewElevenTwelve{}
	\end{quote}

	\begin{quote}
		\BibleLukeSixteenSixteen{}
	\end{quote}

	What is this kingdom people fight to get into since John the
	Baptist?

	\begin{quote}
		\BibleDanielTwoThirtyFive{}
	\end{quote}

	Was the kingdom of God the mountain or the stone? It was both.
	The mountain is the stone, and the stone is the mountain. It
	is just different phases of the \textbf{same} kingdom. Both things
	are different phases of the kingdom of God.

	\begin{quote}
		\BibleJohnThreeOne{}
		\BibleJohnThreeTwo{}
		\BibleJohnThreeThree{}
		\BibleJohnThreeFour{}
		\BibleJohnThreeFive{}
		\BibleJohnThreeSix{}
		\BibleJohnThreeSeven{}
		\BibleJohnThreeEight{}
		\BibleJohnThreeNine{}
		\BibleJohnThreeTen{}
		\BibleJohnThreeEleven{}
		\BibleJohnThreeTwelve{}
	\end{quote}

	Nicodemus asks ``How can these things be?''. Christ said that
	we speak that we do know, and testify what we have seen. Who
	did Christ include in this ``we''? Obviously himself. John the
	Baptist too. Why should Nicodemus understand what he is saying?
	It seems that a teacher should be tapped into the same sources
	that Christ himself is tapped into.

	\begin{quote}
		\BibleJohnThreeTwo{}
		\BibleJohnThreeThree{}
	\end{quote}

	Nicodemus recognized Christ was sent from God. God is with the
	Christ apparently. Chirst's teaching and works testify this.
	Why does Jesus respond to Nicodemus the way he does? Nicodemus
	is sneaking to see the Chirst. Nicodemus sees something in the
	Chirst and his work, but he does not understand what he is 
	seeing.

	\vspace{5mm}
	Strongs G3702: To see

	Thayers:
	\begin{enumerate}
		\item To see with the eyes
		\item To see with the mind; to percieve; to look at,
			to give attention to
		\item To see; i.e., to become aquiainted with by
			experience
	\end{enumerate}

	Which definition did Jesus use with Nicodemus? \textbf{All of
	them.} Jesus was not condemning Nicodemus for not knowing. He
	was recognizing that the leadership of the nation did not
	understand the kingdom of God like Christ did. Nicodemus
	saw that something was going on, but he could not understand.
	So Christ puts him on some game. Nicodemus percieves that Chirst
	has something he needs.

	Zaydok has no problem with Nicodemus coming at night, because
	Zaydok would have probably came by night. Do not pretend that you
	are some apostle and ultra righteous disciple if you were not!
	I probably woulda been creeping in at night aswell --- if
	I had the courage to even do that. Nicodemus needed to see, and
	Nicodemus also was trying to protect what he already had.

	\begin{quote}
		\BibleJohnThreeEighteen{}
		\BibleJohnThreeNineteen{}
		\BibleJohnThreeTwenty{}
		\BibleJohnThreeTwentyOne{}
	\end{quote}

	Nicodemus was coming to the light, even though he was coming
	by night. Chirst exposed Nicodemus to Nicodemus. Christ showed
	him to himself. The word of God exposed me to me. We repented
	for our own ways. No matter what Nicodemus was short in, it's ok,
	because he was drawn to the light.

	\begin{quote}
		\BibleFirstPeterOneSeventeen{}
		\BibleFirstPeterOneEighteen{}
		\BibleFirstPeterOneNineteen{}
		\BibleFirstPeterOneTwenty{}
		\BibleFirstPeterOneTwentyOne{}
		\BibleFirstPeterOneTwentyTwo{}
		\BibleFirstPeterOneTwentyThree{}
		\BibleFirstPeterOneTwentyFour{}
		\BibleFirstPeterOneTwentyFive{}
	\end{quote}

	God's word endures forever. The corruptable things are not
	necessarily bad, they are just not going to get you what you
	are looking for. The traditions of the Fathers will not get
	you the kingdom of God. Since Nicodemus is a teacher of the
	tradition of the Fathers, Nicodemus will not even see the
	kingdom of God if he is not reborn.

	What was Nicodemus teaching in Israel? Whatever it was, it
	does not mean that he understood the kingdom of God. Until
	Nicodemus understands the Son of God, he cannot enter into 
	the kingdom. You must know where the door of the kingdom
	is to get into the kingdom.

	\begin{quote}
		\BiblePsalmThirtyTwoOne{}
		\BiblePsalmThirtyTwoTwo{}
	\end{quote}

	Whoever ignores the Chirst cannot enter the kingdom of God.
	They are condemned already, and have no understanding. They
	are religious, but doomed.

	Was the kingdom of God (as a concept) even on the table before
	Yeshua? No, just a shadow. The kingdom begins when the stone
	hits the image in the feet.

	If Christ is the only door into the kingdom, how do those who
	lived before the son came into the world enter the kingdom?
	Everyone shall stand before the judgement seat of Christ.
	
	Being blameless in the law means does not mean living without
	sin. Paul claimed to be blameless in the law. Whatever you think
	blameless means, it must apply to Paul.

	\begin{quote}
		\BiblePhilippiansThreeTwo{}
		\BiblePhilippiansThreeThree{}
		\BiblePhilippiansThreeFour{}
		\BiblePhilippiansThreeFive{}
		\BiblePhilippiansThreeSix{}
	\end{quote}

	Paul was \textit{blameless}.

	\begin{quote}
		\BibleRomansSevenSeven{}
		\BibleRomansSevenEight{}
		\BibleRomansSevenNine{}
		\BibleRomansSevenTen{}
	\end{quote}

	If there is sin in the nation, what does the day of atonement
	do for the nation? If you got a foul on the nations record,
	the day of atonement is a process (set up by God) to remove the
	foul from your record. The ritual shows that as much as the sin
	is a reality, your obedience and willingness to make amends
	with God according to his law makes you blameless. If you are
	obedient to the commandment, despite previous sin, a person
	can be blameless.

	A woman's menstral period makes her unclean. But by following
	the law of cleaning, she can be purified before God. She
	would be blameless because she follows the law, despite her
	unclean status.

	\begin{quote}
		\BibleSecondCorinthiansFiveNine{}
		\BibleSecondCorinthiansFiveTen{}
	\end{quote}

	All appear before Christ's judgement seat. Even those who did
	not see Chirst in the flesh. This would include us, as well as
	those who came before the Messiah's time. Everyone will be
	judged out of the Lamb's book of life.

	What is the purpose of the Gospel if those who do not hear
	it can be found righteous before Christ? Did the Israelites
	fully understand the meaning behind their rituals? Probably not.

	\begin{quote}
		\BibleHebrewsNineTwentyTwo{}
		\BibleHebrewsNineTwentyThree{}
		\BibleHebrewsNineTwentyFour{}
		\BibleHebrewsNineTwentyFive{}
		\BibleHebrewsNineTwentySix{}
		\BibleHebrewsNineTwentySeven{}
		\BibleHebrewsNineTwentyEight{}
	\end{quote}

	Some Israelites lived under the rituals that revealed the Son
	of God, but died not completely understanding. What hope do they
	have? All men die because of sin. If Christ came to put away sin
	and get victory over death, then we should not have to die anymore!
	God designed the system this way, and he will have mercy on who
	he wants.

	\begin{quote}
		\BibleGenesisSixEight{}
	\end{quote}

	\begin{quote}
		\BibleExodusThirtyThreeNineteen{}
	\end{quote}

	The scriptures say they died in hope. Hope in what? That God
	would justify those he deemed ``good''.

	Doing what you are told, even after sin, is just as valid as
	the sin. 

	God has seen righteousness outside the law.

	\begin{quote}
		\BibleGalatiansFiveThirteen{}
		\BibleGalatiansFiveFourteen{}
	\end{quote}

	\begin{quote}
		\BibleIsaiahTwentyEightFourteen{}
		\BibleIsaiahTwentyEightFifteen{}
		\BibleIsaiahTwentyEightSixteen{}
	\end{quote}


\end{document}
